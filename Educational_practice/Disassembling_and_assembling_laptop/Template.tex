\documentclass[a4paper,14pt]{article}
\usepackage[utf8]{inputenc}
\usepackage[russian]{babel}
\usepackage{geometry}
\geometry{left=3cm, right=2cm, top=2cm, bottom=2cm}
\usepackage{graphicx}
\usepackage{titlesec}
\usepackage{enumitem}
\usepackage{float}
\usepackage{caption}
\usepackage{mathptmx} % Times New Roman для pdflatex

% ВАЖНО: НАСТРОЙКА ПОДПИСЕЙ РИСУНКОВ В ФОРМАТЕ "Рисунок X - Описание"
\captionsetup{
    labelsep=endash,
    justification=centering,
    font=small,
    labelfont=md
}
\addto\captionsrussian{\renewcommand{\figurename}{Рисунок}}

% Форматирование разделов и подразделов
\titleformat{\section}{\normalfont\Large\bfseries}{\thesection}{1em}{}
\titleformat{\subsection}{\normalfont\large\bfseries}{\thesubsection}{1em}{}

% Настройка отступов в списках
\setlist[enumerate]{leftmargin=*}
\setlist[itemize]{leftmargin=*}

\begin{document}

% ТИТУЛЬНЫЙ ЛИСТ
\begin{center}
\textbf{ОТЧЕТ} \\
\vspace{0.5cm}
По учебной практике \\
\vspace{4cm}
\end{center}

\begin{flushright}
Выполнили: ст. гр. ИУ6-13Б Иванов И. И. \\
\end{flushright}

\vspace{16cm}

\begin{center}
г. Москва \\
\vspace{1cm}
2025
\end{center}

\newpage

% СОДЕРЖАНИЕ
\section*{Содержание}
\begin{enumerate}
\item Введение
\item Название основного раздела
    \begin{enumerate}
    \item[2.1.] Название подраздела первого уровня
    \item[2.2.] Название подраздела первого уровня
    \item[2.3.] Название подраздела первого уровня
    \end{enumerate}
\item Название раздела
\item Заключение
\item Используемая литература
\end{enumerate}

\newpage

% ОСНОВНОЕ СОДЕРЖАНИЕ
\section{Введение}

Целью данной учебной практики являлось приобретение практических навыков по обслуживанию и ремонту аппаратного обеспечения. В ходе работы были изучены внутреннее устройство оборудования, методы его безопасной разборки и последующей сборки.

\section{Название основного раздела}

\subsection{Название подраздела первого уровня}

Текст описания этапа работы. Подробное техническое описание выполняемых действий с использованием профессиональной терминологии.

% ПРИМЕР ВСТАВКИ РИСУНКА С ПРАВИЛЬНОЙ ПОДПИСЬЮ
\begin{figure}[H]
\centering
\includegraphics[width=0.6\textwidth]{example_image.jpg}
\caption{Процесс выполнения технологической операции}
\end{figure}

\subsection{Название подраздела первого уровня}

Текст описания этапа работы. Подробное техническое описание выполняемых действий.

\begin{figure}[H]
\centering
\includegraphics[width=0.5\textwidth]{component_view.jpg}
\caption{Вид компонента с подключенными разъемами}
\end{figure}

\section{Сборка устройства}

Сборка устройства производится в последовательности, обратной разборке.

\begin{enumerate}
\item \textbf{Первый этап сборки}. Подробное описание действий.
\item \textbf{Второй этап сборки}. Подробное описание действий.
\item \textbf{Третий этап сборки}. Подробное описание действий.
\end{enumerate}

\begin{figure}[H]
\centering
\includegraphics[width=0.7\textwidth]{assembly_process.jpg}
\caption{Процесс сборки компонентов устройства}
\end{figure}

\section{Заключение}

В ходе учебной практики были успешно освоены основные принципы разборки и сборки оборудования. Были получены практические навыки по аккуратной работе с компонентами.

\section{Используемая литература}

\begin{enumerate}
\item Название книги. – Город: Издательство, год.
\item Название статьи // Название журнала. – год. – № номер. – С. страницы.
\item Электронный ресурс: URL: ссылка (дата обращения: дд.мм.гггг).
\end{enumerate}

\end{document}